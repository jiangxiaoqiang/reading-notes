\documentclass[letter]{book}

%%%%%% Import Package %%%%%%
\usepackage{graphicx}
\usepackage[unicode]{hyperref}
\usepackage{cite}
\usepackage{indentfirst}
\usepackage{multirow}
\usepackage{indentfirst}
\usepackage{titlesec}
\usepackage{xcolor}
\usepackage{listings}
\usepackage{fontspec,xunicode,xltxtra}
\usepackage{xeCJK}
\usepackage{hyperref}
\usepackage{enumerate}
\usepackage{epigraph}
\usepackage{amsmath}
\usepackage[xindy]{glossaries}
\usepackage{fancyhdr}
\usepackage{amsmath}
%\usepackage{TikZ}
\usepackage{ifthen}
\usepackage{longtable}
\usepackage{fontspec}

%%%% 下面的命令设置行间距与段落间距 %%%%
\linespread{1.4}
% \setlength{\parskip}{1ex}
\setlength{\parskip}{0.5\baselineskip}

%Set Code Format%
\lstloadlanguages{C, csh, make,python,Java}
\lstset{	  
	alsolanguage= XML,  
	tabsize=4, %  
	frame=shadowbox, %把代码用带有阴影的框圈起来  
	commentstyle=\color{red!50!green!50!blue!50},%浅灰色的注释  
	rulesepcolor=\color{red!20!green!20!blue!20},%代码块边框为淡青色  
	keywordstyle=\color{blue!90}\bfseries, %代码关键字的颜色为蓝色,粗体  
	showstringspaces=false,%不显示代码字符串中间的空格标记  
	stringstyle=\ttfamily, % 代码字符串的特殊格式  
	keepspaces=true, %  
	breakindent=22pt, %  
	numbers=left,%左侧显示行号 往左靠,还可以为right,或none,即不加行号  
	stepnumber=1,%若设置为2,则显示行号为1,3,5,即stepnumber为公差,默认stepnumber=1  
	%numberstyle=\tiny, %行号字体用小号  
	numberstyle={\color[RGB]{0,192,192}\tiny} ,%设置行号的大小,大小有tiny,scriptsize,footnotesize,small,normalsize,large等  
	numbersep=8pt,  %设置行号与代码的距离,默认是5pt  
	basicstyle=\footnotesize, % 这句设置代码的大小  
	showspaces=false, %  
	flexiblecolumns=true, %  
	breaklines=true, %对过长的代码自动换行  
	breakautoindent=true,%  
	breakindent=4em, %  	   
	aboveskip=1em, %代码块边框  
	tabsize=4,  
	showstringspaces=false, %不显示字符串中的空格  
	backgroundcolor=\color[RGB]{245,245,244},   %代码背景色  
	%backgroundcolor=\color[rgb]{0.91,0.91,0.91}    %添加背景色  
	escapeinside=``,  %在``里显示中文  
	%% added by http://bbs.ctex.org/viewthread.php?tid=53451  
	fontadjust,  
	captionpos=t,  
	framextopmargin=2pt,framexbottommargin=2pt,abovecaptionskip=-3pt,belowcaptionskip=3pt,  
	xleftmargin=4em,xrightmargin=4em, % 设定listing左右的空白  
	texcl=true
}



\begin{document}

\section*{字}



\subsection*{生字}


\begin{tabular}{ccp{8cm}c}
	\hline
	\multirow{1}{*}{字}
	& \multicolumn{1}{c}{读音} 
	& \multicolumn{1}{c}{释义}\\			
		笹    & t\`{i}(四声)       & 矮而细小的竹子。叶常用来保存食物。种类很多。笹是汉语词语,是同“屉”,在日本是指祭祀时用来参拜愿望用的小竹。           \\
	虬 & q\'{i}u(二声)  &古代传说中有角的小龙:虬龙。拳曲:虬曲(盘绕弯曲)。虬须。虬髯(拳曲的胡须,特指两腮上的胡须)。\\
	痄 & zh\`{a}(四声) &  〔痄腮〕一种传染病,耳朵下面肿胀疼痛,病原体是一种滤过性病毒。亦称“流行性腮腺炎” \\
		髡  & k\={u}n(一声) & 古代剃去男子头发的一种刑罚:髡首(剃去头发,光头)。髡钳(剃去头发,并用铁圈束颈)。 \\
		塚 & 冢的旧体字 \\
	\hline
\end{tabular}



\subsection{生词}

\begin{tabular}{ccp{8cm}c}
	\hline
	\multirow{1}{*}{字}
	& \multicolumn{1}{c}{读音} 
	& \multicolumn{1}{c}{释义}\\			
	付之阙如    &       & 付之阙如指缺少某些应该有而没有的。一般词典不收录,属于极偏词汇。         \\
	\hline
	頫 & & 赵孟頫(fǔ)(1254年10月20日[1]  —1322年7月30日[1]  ),字子昂,汉族,号松雪道人 ,又号水晶宫道人、鸥波,中年曾署孟俯。浙江吴兴(今浙江湖州)人。南宋末至元初著名书法家、画家、诗人,宋太祖赵匡胤十一世孙、秦王赵德芳嫡派子孙。\\
	修禊(x\`{i}) & & 修禊,古代传统民俗。季春时,官吏及百姓都到水边嬉游,是古已有之的消灾祈福仪式,后来演变成中国古代诗人雅聚的经典范式,其中以发生在晋唐会稽郡山阴城(今绍兴越城区)的兰亭修禊和长安曲江修禊最为著名。\\
\end{tabular}


\section{词}


\paragraph{奥斯曼帝国}

奥奥斯曼帝国(土耳其语:Osmanlı İmparatorluğu)为奥斯曼人建立的帝国,创立者为奥斯曼一世。奥斯曼人初居中亚,后迁至小亚细亚,日渐兴盛。极盛时势力达亚欧非三大洲,领有南欧、巴尔干半岛、中东及北非之大部分领土,西达直布罗陀海峡,东抵里海及波斯湾,北及今之奥地利和斯洛文尼亚,南及今之苏丹与也门。自消灭东罗马帝国后,定都于君士坦丁堡,且以东罗马帝国的继承人自居。故奥斯曼帝国的君主苏丹以自封的形式,视自己为天下之主,继承了东罗马帝国的文化及伊斯兰文化,因而东西文明在其得以统合。


\paragraph{拜占庭帝国}

东罗马帝国是一个历史上知名的帝国。罗马帝国自东西分治后,帝国东部罗马政权的延续被称为东罗马帝国(相对于帝国西部的西罗马帝国),16世纪以后,开始有学者称之为拜占庭帝国。其国民在其一千多年的存在期内自称为“罗马帝国”(拉丁语:Imperium Romanum;希腊语:Βασιλεία Ρωμαίων)。帝国位于欧洲东部,领土曾包括欧亚非三大洲的亚洲西部和非洲北部,是古代和中世纪欧洲历史上最悠久的君主制国家。


\paragraph{美杜莎}

美杜莎(希腊语:Μέδουσα,又译梅杜莎、墨杜萨),是希腊神话中的一个女妖,戈耳工三姐妹之一,居住在大洋俄刻阿诺斯的彼岸与黑夜之地相接的地方。父亲为大地盖亚与海洋蓬托斯之子福耳库斯,母亲为前两者之女,福耳库斯的姊妹刻托。根据诗人奥维德的《变形记》(Metamorphoses4.770)所述,美杜莎原是一位美丽的少女,因为与海神波塞冬私自约会(也有一些版本称因美杜莎自恃长得美丽,竟然不自量力地和智慧女神比起美来,而被雅典娜诅咒),雅典娜一怒之下将美杜莎的头发变成毒蛇,而且给她施以诅咒,任何直望美杜莎双眼的人都会变成石像,因此成了面目丑陋的怪物。

\paragraph{十字军东征}

十字军东征(拉丁文:Cruciata,1096年-1291年)是一系列在罗马天主教教皇的准许下进行的、持续近200年的、有名的宗教性军事行动,由西欧的封建领主和骑士,对地中海东岸的国家,以收复阿拉伯入侵占领的土地名义发动的战争。前后共计有八次。当时原属于罗马天主教圣地的耶路撒冷落入伊斯兰教手中,罗马天主教为了收复失地,便进行多次东征行动。但实际上东征不仅仅限于针对伊斯兰,如第四次十字军东征就是针对信奉东正教的拜占庭帝国。十字军在他们占领的地区建立起了几十个十字军国家,最大的是耶路撒冷王国,此外还有安条克公国,的黎波里国等。自然,它受到了整个天主教世界累世的传诵,众多随军教士及后世的教会编年史家都在竭力记述此役,赞美基督大能,如神迹般传诵。同时,这场战争及其后拉丁东方的建立,更是影响了整个东地中海格局,如一石激起千层浪,受到各方的强烈关切。拜占庭、亚美尼亚、突厥人、阿拉伯人,各种宗教背景、不同地位出身的史家都在著述陈辞,详述此事,以资借鉴反思。

\paragraph{伊斯坦布尔}

伊斯坦布尔(土耳其语:İstanbul)是土耳其经济、文化、金融、新闻、贸易、交通中心,世界著名的旅游胜地,繁华的国际大都市之一。位于巴尔干半岛东端,博斯普鲁斯海峡南口西岸。扼黑海入口,当欧、亚交通要冲,战略地位极为重要。面积5343平方公里,人口1385万(1985)。公元前658年始建在金角湾与马尔马拉海之间的地岬上,称拜占庭。公元330年改建为东罗马帝国首都,改名为君士坦丁堡(Constantinople,又译康斯坦丁堡)。别称新罗马。1453年成为奥斯曼帝国首都。伊斯坦布尔之名在奥斯曼帝国征服之前至少存在百余年历史了,如1403年西班牙国王遣使觐见帖木儿大帝,使臣途径君士坦丁堡,在回忆录中提到,希腊人也称此地为伊斯坦布尔(见《克拉维约东使记》商务印书馆汉译本)。但西方国家认为奥斯曼帝国是此地的侵略者,所以依然坚持称此地为君士坦丁堡。1923年土耳其共和国初建时为首都(独立战争期间迁都安卡拉),伊斯坦布尔才成为国际上的正式名称。现在市区已扩大到金角湾以北,博斯普鲁斯海峡东岸的子斯屈达尔也划入市区,成为地跨欧、亚两洲的现代化城市。
伊斯坦布尔当选为2010年欧洲文化之都和2012年欧洲体育之都。该市的历史城区在1985年被联合国教科文组织列为世界遗产。

\end{document}
\theend
